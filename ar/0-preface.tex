\documentclass[arabic]{article}

\usepackage[bidi=basic, english]{babel}
\babelprovide[main, import]{arabic}
\babelprovide[import]{english}

\usepackage{fontspec}

\setmainfont{NotoSansArabic}[
  Script=Arabic,
  Scale=1.0
]

\babelfont[english]{rm}[Scale=1.0]{Noto Sans}

\begin{document}

\begin{center}
بِسْمِ ٱللَّهِ ٱلرَّحْمَٰنِ ٱلرَّحِيمِ
\end{center}

خير ما يستهل به كل قول، الحمد لله رب العالمين، وأفضل ما قفي به الحمد الصلاة والسلام على رسول الله أشرف المرسلين، الرسول العربي الأمي المبعوث رحمة للعالمين، اللهم صلي وسلم عليه وعلى آله وصحبه أجمعين. \\

سلسلة "مفاتيح علوم الحاسوب" هي مجموعة من المقولات، كل مقولة تحتوي على منفعة علمية بعلوم الحاسوب بشيء من الاستقال؛ أي أن كل مقولة مستقلة بذاتها لا تعتمد على مقولات متقدمة أو متأخرة لتحصيل مجمل الفهم، فلا تتطلب من قارئها حيد النظر إلى غيرها. \\

صفة الإستقلال تلك لا تعني بالضرورة انعدام الترابط بين المقولات؛ فجميع المقولات تربطها الغاية في تحصيل علوم الحاسوب، وبضعها يربط الآخر في "الخوارزميات"، وبعضها في "تعلم الآلة"، والبعض الآخر في "تصميم التطبيقات"... الخ. فستجد المقولات تشير إلى بعضها البعض ليس كنوع من الإعتمادية وإنما لتحصيل كامل الفهم بعد مجمله. \\

كل مقولة صنفت وؤلفت لتكون محكمة الكلم، بسيطة الفهم، ماتعة للذهن؛ فلا يُشتت قارئها في معانيها، ولا يُحير في تحصيل مفاهيمها، ولا يَمل من قراءتها؛ ولذلك صبا العبد الفقير لتنصيفها بالعربية فضلا عن العامية الدارجة كما هو معهود الآن على منصات التواصل. \\

سميت "مفاتيح" اقتداءا بكتاب "مفاتيح العلوم" للجد الأكبر، محمد بن موسى الخوارزمي — رحمه الله، و"علوم" لتفرع وكثرة العلوم والفنون الداخلة في السلسلة فلا تقتصر على التكلم في علم بذاته من العلوم المتعلقة بالآلة، و"الحاسوب" ترجمة للفظ الإنجليزي "Computer" وقد فضلنا استعمال وزن الآلة "فاعول"، على غير المعهود، لدقة دلالته. \\

ختاما، لا ندعي كامل العلم والمعرفة، وإنما نشارك ما تفضل الله به علينا من فهم وعلم مع الناس أجمعين، ونسأله تعالى أن يوفق خطانا وأن يزيدنا علما وأن يجعل في عملنا هذا المنفعة لعباده الصالحين.

\end{document}
